% $Log: abstract.tex,v $
% Revision 1.1  93/05/14  14:56:25  starflt
% Initial revision
% 
% Revision 1.1  90/05/04  10:41:01  lwvanels
% Initial revision
% 
%
%% The text of your abstract and nothing else (other than comments) goes here.
%% It will be single-spaced and the rest of the text that is supposed to go on
%% the abstract page will be generated by the abstractpage environment.  This
%% file should be \input (not \include 'd) from cover.tex.
While aerobic respiration is typically invoked as the dominant mass-balance sink for organic matter in the upper ocean, many other biological and abiotic processes can degrade particulate and dissolved substrates on globally significant scales. The relative strengths of these other remineralization processes --- including mechanical mechanisms such as dissolution and disaggregation of sinking particles, and abiotic processes such as photooxidation --- remain poorly constrained. In this thesis, I examine the biogeochemical significance of various alternative pathways of organic matter remineralization using a combination of field experiments, modeling approaches, geochemical analyses, and a new, high-throughput lipidomics method for identification of lipid biomarkers. I first assess the relative importance of particle-attached microbial respiration compared to other processes that can degrade sinking marine particles. A hybrid methodological approach --- comparison of substrate-specific respiration rates from across the North Atlantic basin with Monte Carlo-style sensitivity analyses of a simple mechanistic model --- suggested sinking particle material was transferred to the water column by various biological and mechanical processes nearly 3.5 times as fast as it was directly respired, questioning the conventional assumption that direct respiration dominates remineralization. I next present and demonstrate a new lipidomics method and open-source software package for discovery and identification of molecular biomarkers for organic matter degradation in large, high-mass-accuracy HPLC-ESI-MS datasets. I use the software to unambiguously identify more than 1,100 unique lipids, oxidized lipids, and oxylipins in data from cultures of the marine diatom \emph{Phaeodactylum tricornutum} that were subjected to oxidative stress. Finally, I present the results of photooxidation experiments conducted with liposomes --- nonliving aggregations of lipids --- in natural waters of the Southern Ocean. A broadband polychromatic apparent quantum yield (AQY) is applied to estimate rates of lipid photooxidation in surface waters of the West Antarctic Peninsula, which receive seasonally elevated doses of ultraviolet radiation as a consequence of anthropogenic ozone depletion in the stratosphere. The mean daily rate of lipid photooxidation (50 $\pm$ 11 pmol IP-DAG L$^{-1}$ d$^{-1}$, equivalent to 31 $\pm$ 7 $\mu$g C m$^{-3}$ d$^{-1}$) represented between 2 and 8 \% of the total bacterial production observed in surface waters immediately following the retreat of the sea ice.
