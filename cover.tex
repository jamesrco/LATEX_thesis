% -*-latex-*-
% 
% For questions, comments, concerns or complaints:
% thesis@mit.edu
% 
%
% $Log: cover.tex,v $
% Revision 1.8  2008/05/13 15:02:15  jdreed
% Degree month is June, not May.  Added note about prevdegrees.
% Arthur Smith's title updated
%
% Revision 1.7  2001/02/08 18:53:16  boojum
% changed some \newpages to \cleardoublepages
%
% Revision 1.6  1999/10/21 14:49:31  boojum
% changed comment referring to documentstyle
%
% Revision 1.5  1999/10/21 14:39:04  boojum
% *** empty log message ***
%
% Revision 1.4  1997/04/18  17:54:10  othomas
% added page numbers on abstract and cover, and made 1 abstract
% page the default rather than 2.  (anne hunter tells me this
% is the new institute standard.)
%
% Revision 1.4  1997/04/18  17:54:10  othomas
% added page numbers on abstract and cover, and made 1 abstract
% page the default rather than 2.  (anne hunter tells me this
% is the new institute standard.)
%
% Revision 1.3  93/05/17  17:06:29  starflt
% Added acknowledgements section (suggested by tompalka)
% 
% Revision 1.2  92/04/22  13:13:13  epeisach
% Fixes for 1991 course 6 requirements
% Phrase "and to grant others the right to do so" has been added to 
% permission clause
% Second copy of abstract is not counted as separate pages so numbering works
% out
% 
% Revision 1.1  92/04/22  13:08:20  epeisach

% NOTE:
% These templates make an effort to conform to the MIT Thesis specifications,
% however the specifications can change.  We recommend that you verify the
% layout of your title page with your thesis advisor and/or the MIT 
% Libraries before printing your final copy.

\title{The Remineralization of Marine Organic Matter by Diverse Biological and Abiotic Processes}

\author{James R. Collins}
% If you wish to list your previous degrees on the cover page, use the 
% previous degrees command:
%       \prevdegrees{A.A., Harvard University (1985)}
% You can use the \\ command to list multiple previous degrees
       \prevdegrees{B.A., Yale College, 2004 \\
                    M.E.Sc., Yale School of Forestry \& Environmental Studies, 2011}
\department{Joint Program in Oceanography\\Massachusetts Institute of Technology \& Woods Hole Oceanographic Institution}

% If the thesis is for two degrees simultaneously, list them both
% separated by \and like this:
 \degree{Doctor of Philosophy}
%\degree{Bachelor of Science in Computer Science and Engineering}

% As of the 2007-08 academic year, valid degree months are September, 
% February, or June.  The default is June.
\degreemonth{February}
\degreeyear{2017}
\thesisdate{January 20, 2017}

%% By default, the thesis will be copyrighted to MIT.  If you need to copyright
%% the thesis to yourself, just specify the `vi' documentclass option.  If for
%% some reason you want to exactly specify the copyright notice text, you can
%% use the \copyrightnoticetext command.  
%\copyrightnoticetext{\copyright IBM, 1990.  Do not open till Xmas.}
\copyrightnoticetext{\copyright 2017 James R. Collins.  All rights reserved. 
\\The author hereby grants to MIT and WHOI permission to reproduce and 
to distribute publicly paper and electronic copies of this thesis document 
in whole or in part in any medium now known or hereafter created.}

% If there is more than one supervisor, use the \supervisor command
% once for each.
\supervisor{Dr. Benjamin A. S. Van Mooy}{Senior Scientist, Woods Hole Oceanographic Institution}


%Henrik Schmidt
%Chairman, Joint Committee for Applied Ocean Science & Engineering
%Massachusetts Institute of Technology
%Woods Hole Oceanographic Institutio
\chairwhoi{Prof. Shuhei Ono}{Associate Professor of Geochemistry, MIT\linebreak Chair, Joint Committee for Chemical Oceanography}

% Make the titlepage based on the above information.  If you need
% something special and can't use the standard form, you can specify
% the exact text of the titlepage yourself.  Put it in a titlepage
% environment and leave blank lines where you want vertical space.
% The spaces will be adjusted to fill the entire page.  The dotted
% lines for the signatures are made with the \signature command.
\maketitle

% The abstractpage environment sets up everything on the page except
% the text itself.  The title and other header material are put at the
% top of the page, and the supervisors are listed at the bottom.  A
% new page is begun both before and after.  Of course, an abstract may
% be more than one page itself.  If you need more control over the
% format of the page, you can use the abstract environment, which puts
% the word "Abstract" at the beginning and single spaces its text.

%% You can either \input (*not* \include) your abstract file, or you can put
%% the text of the abstract directly between the \begin{abstractpage} and
%% \end{abstractpage} commands.

% First copy: start a new page, and save the page number.
\cleardoublepage
% Uncomment the next line if you do NOT want a page number on your
% abstract and acknowledgments pages.
% \pagestyle{empty}
\setcounter{savepage}{\thepage}
\begin{abstractpage}
\currentpdfbookmark{Abstract}{Abstract}
% $Log: abstract.tex,v $
% Revision 1.1  93/05/14  14:56:25  starflt
% Initial revision
% 
% Revision 1.1  90/05/04  10:41:01  lwvanels
% Initial revision
% 
%
%% The text of your abstract and nothing else (other than comments) goes here.
%% It will be single-spaced and the rest of the text that is supposed to go on
%% the abstract page will be generated by the abstractpage environment.  This
%% file should be \input (not \include 'd) from cover.tex.
While aerobic respiration is typically invoked as the dominant mass-balance sink for organic matter in the upper ocean, many other biological and abiotic processes can degrade particulate and dissolved substrates on globally significant scales. The relative strengths of these other remineralization processes --- including mechanical mechanisms such as dissolution and disaggregation of sinking particles, and abiotic processes such as photooxidation --- remain poorly constrained. In this thesis, I examine the biogeochemical significance of various alternative pathways of organic matter remineralization using a combination of field experiments, modeling approaches, geochemical analyses, and a new, high-throughput lipidomics method for identification of lipid biomarkers. I first assess the relative importance of particle-attached microbial respiration compared to other processes that can degrade sinking marine particles. A hybrid methodological approach --- comparison of substrate-specific respiration rates from across the North Atlantic basin with Monte Carlo-style sensitivity analyses of a simple mechanistic model --- suggested sinking particle material was transferred to the water column by various biological and mechanical processes nearly 3.5 times as fast as it was directly respired, questioning the conventional assumption that direct respiration dominates remineralization. I next present and demonstrate a new lipidomics method and open-source software package for discovery and identification of molecular biomarkers for organic matter degradation in large, high-mass-accuracy HPLC-ESI-MS datasets. I use the software to unambiguously identify more than 1,100 unique lipids, oxidized lipids, and oxylipins in data from cultures of the marine diatom \emph{Phaeodactylum tricornutum} that were subjected to oxidative stress. Finally, I present the results of photooxidation experiments conducted with liposomes --- nonliving aggregations of lipids --- in natural waters of the Southern Ocean. A broadband polychromatic apparent quantum yield (AQY) is applied to estimate rates of lipid photooxidation in surface waters of the West Antarctic Peninsula, which receive seasonally elevated doses of ultraviolet radiation as a consequence of anthropogenic ozone depletion in the stratosphere. The mean daily rate of lipid photooxidation (50 $\pm$ 11 pmol IP-DAG L$^{-1}$ d$^{-1}$, equivalent to 31 $\pm$ 7 $\mu$g C m$^{-3}$ d$^{-1}$) represented between 2 and 8 \% of the total bacterial production observed in surface waters immediately following the retreat of the sea ice.

\end{abstractpage}

% Additional copy: start a new page, and reset the page number.  This way,
% the second copy of the abstract is not counted as separate pages.
% Uncomment the next 6 lines if you need two copies of the abstract
% page.
% \setcounter{page}{\thesavepage}
% \begin{abstractpage}
% % $Log: abstract.tex,v $
% Revision 1.1  93/05/14  14:56:25  starflt
% Initial revision
% 
% Revision 1.1  90/05/04  10:41:01  lwvanels
% Initial revision
% 
%
%% The text of your abstract and nothing else (other than comments) goes here.
%% It will be single-spaced and the rest of the text that is supposed to go on
%% the abstract page will be generated by the abstractpage environment.  This
%% file should be \input (not \include 'd) from cover.tex.
While aerobic respiration is typically invoked as the dominant mass-balance sink for organic matter in the upper ocean, many other biological and abiotic processes can degrade particulate and dissolved substrates on globally significant scales. The relative strengths of these other remineralization processes --- including mechanical mechanisms such as dissolution and disaggregation of sinking particles, and abiotic processes such as photooxidation --- remain poorly constrained. In this thesis, I examine the biogeochemical significance of various alternative pathways of organic matter remineralization using a combination of field experiments, modeling approaches, geochemical analyses, and a new, high-throughput lipidomics method for identification of lipid biomarkers. I first assess the relative importance of particle-attached microbial respiration compared to other processes that can degrade sinking marine particles. A hybrid methodological approach --- comparison of substrate-specific respiration rates from across the North Atlantic basin with Monte Carlo-style sensitivity analyses of a simple mechanistic model --- suggested sinking particle material was transferred to the water column by various biological and mechanical processes nearly 3.5 times as fast as it was directly respired, questioning the conventional assumption that direct respiration dominates remineralization. I next present and demonstrate a new lipidomics method and open-source software package for discovery and identification of molecular biomarkers for organic matter degradation in large, high-mass-accuracy HPLC-ESI-MS datasets. I use the software to unambiguously identify more than 1,100 unique lipids, oxidized lipids, and oxylipins in data from cultures of the marine diatom \emph{Phaeodactylum tricornutum} that were subjected to oxidative stress. Finally, I present the results of photooxidation experiments conducted with liposomes --- nonliving aggregations of lipids --- in natural waters of the Southern Ocean. A broadband polychromatic apparent quantum yield (AQY) is applied to estimate rates of lipid photooxidation in surface waters of the West Antarctic Peninsula, which receive seasonally elevated doses of ultraviolet radiation as a consequence of anthropogenic ozone depletion in the stratosphere. The mean daily rate of lipid photooxidation (50 $\pm$ 11 pmol IP-DAG L$^{-1}$ d$^{-1}$, equivalent to 31 $\pm$ 7 $\mu$g C m$^{-3}$ d$^{-1}$) represented between 2 and 8 \% of the total bacterial production observed in surface waters immediately following the retreat of the sea ice.

% \end{abstractpage}

\cleardoublepage

%\section*{Biography}
%{\color{Red}
%ADD BIO HERE!!!
%}%end red
\currentpdfbookmark{Acknowledgments}{Acknowledgments}
\begin{singlespace}
\section*{Acknowledgments}

Modern science is a highly collaborative enterprise. And yet --- at times --- it can be incredibly isolating. A great number of people have nurtured, challenged, and supported me scientifically and emotionally over the past five and a half years as I have operated at fits and starts in each of these modes of scientific inquiry. While I've thanked many of these individuals in these Acknowledgments and in the acknowledgments sections of my separate thesis chapters, I am sure I've committed the sin of omission. To those whose names do not appear here, thank you as well.

First, I thank my advisor, Benjamin Van Mooy. A half-decade after arriving in the Joint Program with what I always assumed was a greater-than-usual case of impostor syndrome, I finally feel as though I have become --- under his guidance --- a scientist, geochemist, and oceanographer. Ben's mentorship through the good and the bad was always full of insightful questions, helpful corrections and suggestions, and a brilliant capacity for designing and executing successful experiments in the field and in the laboratory. I also thank (in alphabetical order) the members of my thesis committee: Hugh Ducklow, Philip Gschwend, Colleen Hansel, and Elizabeth Kujawinski. I have come to know each of my committee members in some way beyond the nominal roles they have played in guiding me through my dissertation. As instructors in the classroom, shipmates on research cruises, supervisors in the laboratory, and co-authors on various manuscripts, these talented women and men have taught me so very much. I am indebted in particular to Hugh, whose brilliance, mentorship, kindness, encouragement, and willingness to take me on as a volunteer with the Palmer LTER study carried me through some of the most difficult phases of my dissertation, yet provided me with a source of great scientific and personal inspiration. In addition, Bernhard Peucker-Ehrenbrink served as chair of my thesis defense and assisted me in navigating other challenges I encountered as a student in the Joint Program.

I owe a special debt to Helen Fredricks, Justin Ossolinski, and MK1 Steven Jayne. Helen is a brilliant organic chemist and has been an incredible mentor, collaborator, and surrogate mother to me over the past half-decade. Justin is a man who knows how to get things done. His efforts and know-how have been critical to every one of my scientific endeavors at WHOI and in far-flung places and oceans. Justin has also been a close friend and he generously shared with me his knowledge of the waters in and around Woods Hole. Steve has been an understanding but critical mentor, a good friend, and an unexpected yet much-needed Coast Guard shipmate at WHOI. Steve: You were right, it was all (mostly) okay. I owe much of my accomplishment to other members of the Van Mooy Lab, past and present: Kim Popendorf, Bethanie Edwards, Suni Shah, Patrick Martin, Jamey Fulton (the ``other'' Jamie), Kevin Becker, Jon Hunter, Kyle Mayers, J. Tags, and Fiona Hopewell.

The administrative staff of the WHOI MC\&G Department --- Sheila Clifford, Mary Murphy, Mary Zawoysky, Donna Mortimer, and Linda Cannata --- have been a godsend. I am also grateful for the incredible support I've received from the entire WHOI Academic Programs Office staff and many of the staff at MIT EAPS HQ: Julia Westwater, Lea Fraser, Tricia Morin Gebbie, Christine Charette, Meg Tivey, Jim Yoder, Valerie Caron, Linda Cannata, Ronni Schwartz, Kris Kipp, and Roberta Allard.

For various scientific discussions, collaborations, and inspiration, I also thank Krista Longnecker, Scott Doney, Ollie Zafiriou, Dan Repeta, Amanda Spivak, Paul Fucile, Pete Raymond, and the many instructors of my courses at WHOI, MIT, and the Yale School of Forestry \& Environmental Studies. The science in this thesis would not exist without the ships and research facilities in which I have embarked over the past five and half years. Of course, ships are only as good as the people who man them and I therefore thank the crews (past and present) of the R/V \emph{Knorr}, ARSV \emph{Laurence M. Gould}, R/V \emph{Ka`imikai-O-Kanaloa}, R/V \emph{Clifford A. Barnes} (the former USCGC \emph{Bitt}), SSV \emph{Corwith Cramer}, and the staff of Palmer Station.

My JP cohort --- particularly my officemate, Winn Johnson --- have been an incredible source of support and (when needed) diversion. I also thank Max Kaplan, Cam Braun, Kate French, Deepak Cherian, Melissa Moulton, Dan Amrhein, Harriet Alexander, Katie Pitz, Sophia Merrifeld, Becca Jackson, Emily Zakem, Jill McDermott, Randie Bundy, and Rene Boiteau, with whom I have shared both good and trying times in Woods Hole and Cambridge.

For their support and friendship in ways both professional and personal, I thank Jo Carey, Steve Brady, David Butman, David Griffith, Cam Moore, Charlie Munford, Dr.\textsuperscript{2} Christopher Bartley, Lauren Brooks, Jeff Bowman, Naomi Shelton, Jeffrey Brodeur, Stace Beaulieu, Dan McCorkle, Valier Galy, Dave Glover, Carl Johnson, Steve Manganini, Phoebe Lam, Kay Bidle, Eugene Melamud, the staff at Bioconductor, Matthew Barton, Tim Silva, Mary Beth Decker \& Paul Turner, S. Marshall Griffin, Patrick Haney, Julia Diaz, Sebastian Vivancos, Tina Haskins, Carolyn Lipke, and Dug, Kona, Cutter, and Sophie.

I also owe a debt to my U.S. Coast Guard shipmates and supervisors at the First Coast Guard District, Sector Southeastern New England, and Sector Boston. These incredibly professional men and women patiently allowed me to continue my Reserve service as I sank ever deeper into my Ph.D. studies, making manageable an at-times very challenging competition between the two major professional obligations in my life.

Finally, I thank my family: My mother and father, for providing me with a life's worth of opportunities, inspiring in me a love of the outdoors and the natural world, and for imparting to me the undying wisdom and curiosity of their respective careers in science and education; my brother Andrew, for his support and friendship; and my partner Meredith, for her love and constant encouragement.

Most of the work in this thesis was supported by awards to my advisor from the National Science Foundation (NSF OCE-1155438, OCE-1059884, and OCE-1031143), the Gordon and Betty Moore Foundation (GBMF3301), the Woods Hole Oceanographic Institution (through a Cecil and Ida Green Foundation Innovative Technology Award), and the Simons Foundation as part of the Simons Collaboration on Ocean Processes and Ecology (SCOPE). My work at Palmer Station and aboard the ARSV \emph{Laurence M. Gould} was supported by the Palmer LTER study (NSF awards OPP-9011927, 9632763, 0217282, 0823101, and GEO-PLR 1440435, to H. Ducklow and others).

I was personally supported in the middle years of my thesis research by a U.S. Environmental Protection Agency (EPA) STAR Graduate Fellowship (Fellowship Assistance agreement FP-91744301-0). During my fourth year, I received support from the Stanley W. Watson Student Fellowship Fund. Benefits I earned on active duty under the Post 9/11 GI Bill supported me financially in the first year of my Ph.D. studies. I received supplementary funding for my work in Antarctica through an award from the WHOI Ocean Ventures Fund. The contents of this thesis have not been formally reviewed by EPA. The views expressed in this thesis are solely mine and those of my co-authors, and EPA does not endorse any products or commercial services mentioned therein.
\end{singlespace}
%%%%%%%%%%%%%%%%%%%%%%%%%%%%%%%%%%%%%%%%%%%%%%%%%%%%%%%%%%%%%%%%%%%%%%
% -*-latex-*-